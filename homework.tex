\documentclass{article}
\usepackage[czech]{babel}
\usepackage[utf8x]{inputenc}
\usepackage[T1]{fontenc}
\usepackage[plainpages=false,pdfpagelabels,unicode]{hyperref}
\begin{document}
\chapter{Přehledová kapitola}
\section{Současný stav}

V současné době je vizualizace rozdělení disků při instalaci systému použita v minimu případů. Dále v kapitole rozeberu jednotlivé ukázky které jsem vybral, avšak souhrnně se dá říct, že instalátory se drží textového seznamu diskových oddílů uspořádaných do stromové struktury. Systémy jsem vybíral tak aby bylo možné porovnat alespoň nějakou vizuální stránku. Proto jsem vynechal příklady typu Archlinux či Gentoo, které používají pouze instalaci z  příkazové řádky.  Dále pak uvedu pár příkladů vizualizace kterou používají nástroje na práci s disky jako je například program gparted.

\section{Příklady instalátorů - Linux}

\section{Debian}

Pro příklad jsem vybral 3 linuxové distribuce. Debian jakožto velmi konzervativní distribuci, udržující osvětčené postupy a programy, snažící se o maximální stabilitu i za cenu zastaralosti. Tato distribuce má grafický instalátor, spoléhá ovšem na zkušenosti a znalosti uživatele. K žádnému schematu se během instalace nedostaneme. Jak je vidět na obrázcích, jediný způsob předání informace o plánovaném stavu disku je textový strom diskových oddílů obohacený o možnost výběru a ovládání myší. 

\section{Ubuntu}

Ubuntu linux je dalším příkladem, ač vychází z výše zmíněné distribuce Debian, instalátor používá svůj vlastní. Je také jediným zástupcem linuxové distribuce která využívá alespoň nějaké schéma pro znázornění stavu rozděleného disku. Dříve využívané schéma programu gparted bylo nahrazeno jednoduchou linkou v horní oblasti okna instalátoru. Na této lince jsou barevně znázorněny diskové oddíly vytvořené uživatelem. Stejné barvy jsou poté použity u každého ze záznamů v seznamu oddílů, jak je možné vidět na obrázku. Tento jednoduchý diagram umožňuje rychlý odhad poměrů různých částí které budou vytvořeny.

\section{CentOS}

Jako příklad systémů které využívají instalátor Anaconda jsem vybral systém CentOS. Tato zkratka znamená Community ENterprise Operating System. Jedná se v podstatě o systém Red Hat Enterprise Linux, ovšem bez podpory a opravných patchů od společnosti Red Hat. V současné době instalátor Anaconda používá též pouze textovou reprezentaci disku. Rozdíl oproti ostatním distribucím tvoří seznam změn který je zobrazen před finálním potvrzením a započetím formátování. Na obrázcích můžeme vidět příklad tohoto seznamu. Situaci zpřehledňuje ale pouze pro malý počet změn. Seznam s 30 záznamy o změnách je nepřehledný. Právě tuto situaci se snažím zlepšit v této práci.

\section{Windows 10}

\section{Programy sloužící pro manipulaci s disky}

\section{Gpatred}

\section{blivet-gui}
\end{document}
